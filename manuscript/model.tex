%        File: model.tex

\documentclass[a4paper]{article}
%\documentclass{article}

% Packages
\usepackage{bm}      % for math bold
\usepackage{amsmath} % for the allignment of equations under same number; matrices
\usepackage[a4paper, total={15cm, 23cm}]{geometry} %a4 = 21cm x 29.7cm

% Layout and formatting
\renewcommand{\baselinestretch}{1.5}

% Document Descriptions
\title{Prior distributions for variance-covariance matrices in hierarchical models}
\author{Edoardo Costantini}
\date{\today}

% Start Document
\begin{document}

\maketitle

\section{Introduction}
Fully-Bayesian analysis has grown in popularity over the last few
decades (look for reference). In particular, fully Bayesian 
approaches to hierarchical models are gaining momentum thanks to
the wider accessibility of computing power (ehh.. verify!).  The field is flourishing
with both in terms of practical applications and theoretical interest.

Among the most active theoretical discussions, there is a long 
lasting debate on the value of subjective versus objective Bayesian 
perspectives (see Berger, 2006).  While the former approach is often 
defended as the more truly "Bayesian", allowing the analyst to
include prior information in the estimation process, the latter is 
appropriate when a researchers wants to express lack of subjective 
information on the model parameters of interest.  The present work 
is not aiming to contribute to this debate directly, but rather is meant to
focus on a particular issue pertaining to the objective framework.

When subscribing to the "objective" framework, a researcher
can reflect the lack of prior information, relating to the parameters, 
by defining prior distributions that are meant to be minimally 
informative in some sense.  Many theoretical contributions have tried 
to indicate what distributions, in which situations, are more apt to 
achieve such a goal.  These endeavors have produced practical guidelines 
for researchers to assist them in the crucial decision prior 
distributions represent in Bayesian analysis.  

In making this decision, two concepts are fundamental: conditional 
conjugacy and informativeness of a prior distribution. A family of 
priors for a parameter is conditionally conjugate when the conditional 
posterior of said parameter is also in that family of distributions.  
An uninformative prior is a reference prior distribution defined in 
such a way that the posterior inference is not influence in any way 
by it.  Related to this concept, is that of a weakly informative prior.  
This is a proper distribution (it integrates to 1) but is set up so 
that "the information it does provide is intentionally weaker than 
whatever actual prior knowledge is available" (Gelman, 2006).  This 
definition is rooted in the convenience, one may even say need, for 
a reference posterior distribution, by which is meant a distribution, 
obtained by employing an improper prior, that approximates a posterior 
that would have been obtained if a proper prior, describing initial 
vague information, had been used (Bernardo, 1979). In this paper, we 
will focus on proper priors, but the principle of uninformativeness 
still applies: objective Bayesian analysis needs a posterior 
distribution that does not incorporate the researcher's personal 
beliefs so that it is possible to asses the relevance of initial 
information.

Much attention has been dedicated in the literature to propose
prior distributions for scale parameters that are alternatives to the
overused inverse Gamma distribution (reference dump here). This effort
finds its \textit{raison d'être} in the undesirable shrinkage of the 
posterior distributions towards zero, even when the parameters of an 
inverse Gamma prior are specified to achieve weak informativeness (Brown 
and Draper, 2006; Gelman, 2006). Many alternatives have been proposed
but the focus has always been on hierarchical models with single random 
effects, even when the declared intent was to speak of cases of higher 
dimensionality (Kass and Natarajan, 2006)

My work contributes to this literature by exploring the degree of 
uninformativity achieved by different prior distributions for the 
covariance matrix of a vector of random effects in a hierarchical
model.

\section{The hierarchical model of interest}

I worked with a two-level normal model of repeated observations $y_{ij}$
with individual (clusters) effects $\bm{b}_i$

\begin{equation}
 \begin{split}
	y_{ij}& = \bm{x}^{T}_{ij} \bm{\theta} + \bm{z}^{T}_{ij}\bm{b}_i + \epsilon_{ij}\\
	\bm{b}_i& \sim N(\bm{0}, \bm{\Psi})\\
	\epsilon_{ij}& \sim N(0, \sigma^2)
 \end{split}
\end{equation}

with $\bm{b}_i$ a vector of ranodm effects, and $\bm{\Psi}$, and 
$\sigma^2$ representing the model hyperparameters.  The model is 
fitted to a repeated observations data set where the main predictor 
is \textit{time} ($x_1$) and there is a dichotomous covariate $x_2$ 
that interacts with it.  $\theta$ is a vector of fixed effects 
containing fixed intercept, time, covariate, and interaction effects.  
This model is a random intercept and random slope model, with $\bm{b}_i
 = [b_{i0}, b_{i1}]'$ a vector containing the cluster (individual) 
specific effects.

\section{Weakly informative prior distributions in hierarchical models}

The weak informativeness of a prior is inherently a provisional concept.
It makes sense to fit a model with a specific prior definition,
judge whether the the posterior distribution makes sense (e.g.
the values allowed by the posterior should be in a plausible range 
for the parameter of interest, and the posterior itself should not
be shrieked to zero by the prior); and adjust the prior specification
if the posterior does not make sense. In other words, if the posterior
obtained using a given prior spreads over an unrealistic range of values
for the parameter (e.g. excessively large variances), or if it is shrieked
towards an implausible small number, then it is reasonable to reconsider
the prior choice.

In the case of scalar random effects, such as for a random intercept
model (as opposed to the vector of random effects that characterizes
random intercept ad slope models), it has been previously shown that 
the Inverse-gamma distribution with parameters $\alpha = \epsilon$, 
$\beta = \epsilon$, commonly considered non informative, does not look
to uphold to such a property (Gelman, 2006).

A fundamental issue of Bayesian Statistics is the definition 
uninformative priors. It is common to distinguish between 
objective and subjective Bayesian approaches to the definition 
of priors (elaborate more).  As Bayesian approaches to data 
analysis become more and more widespread a need for standard 
ways of defining priors for analyses becomes more and more 
relevant. While this issue might be easier for some types of 
analysis (elaborate more), when models become more complicated 
the issue becomes non-trivial.  In particular there is no 
standard way of choosing an uninformative prior for the random 
effects variance-covariance matrix parameter in mixed effects 
models.

Publications have flourished on the matter but have often 
focused on the simpler case of a scalar random effect, the random 
intercept model. Here I will explain Brown and Draper 2006, 
Gelman 2006, Kass and Natarajan 2006 solutions.
Some publications extended the reasoning to the vector random 
effects case, elaborating on the definition of a prior distribution 
for the variance covariance matrix (henceforth referred to as 
$\bm{\Psi}$), as random intercept and random slopes models.

Here I define the Bayesian model I want to use to test the performance 
of the different priors. The features I want to include are the 
following: continuous outcome, any number/measurement scale.

Let's start with the \textbf{model}

\begin{equation}
 \begin{split}
	y_{ij}& = \bm{x}^{T}_{ij} \bm{\theta} + \bm{z}^{T}_{ij}\bm{b}_i + \epsilon_{ij}\\
	\bm{b}_i& \sim N(\bm{0}, \bm{\Psi})\\
	\epsilon_{ij}& \sim N(0, \sigma^2)
 \end{split}
\end{equation}

(Vectors are in bold, matrix are capital Greek letters).\\
I'm going to define the following \textbf{priors}:

\begin{equation}
p(\bm{\theta}) \propto 1
\end{equation}
\begin{equation}
p(\sigma^2) \propto \sigma^{-2}
\end{equation}

For what concerns the random effects variance covariance matrix, 
different priors are tested. In particular we used:

\paragraph{inverse-Wishart prior}

\begin{equation}
p(\bm{\Psi}) \propto IW(\nu, S_{0})
\end{equation}	

where $\bm{\Psi}$ is the matrix of variance-covariance hyperparameters
defined in model 1; $S_{0}$ is a $k \times k$ matrix, usually considered
as a prior guess for the covariance matrix. The diagonal elements of $\bm{\Psi}$
are inverse-Wisahrt distributed as well. In particular, the $k_1 \times 
k_1$ upper-left triangular sub-matrix $\bm{\Psi}_{11}$ is $IW(\nu-(k-k_1), S_{011})$
distributed.  Furthermore, note that in our case the diagonal elements
are the random intercept and slope parameters, and therefore are scalar.  
For a univariate case, the inverse Wishart distribution simplifies 
to an inverse Gamma with parameters $\alpha=\frac{\nu}{2}, \beta = 
\frac{S_{0kk}}{2}$. This means that $\sigma_{i0}^2 \sim IW(\nu - 1, S_{011})$
becomes $\sigma_{i0}^{2} \sim IG(\frac{\nu-1}{2},\frac{S_{0ii}}{2})$.
By choosing $\nu = k - 1 + \epsilon$ and $S_{011} = k - 1 - 1 + \epsilon$
we obtain a prior on the variance parameters equivalent to the one 
$IG(\epsilon, \epsilon)$ studied by Gelman (2006). Note that $k$ being 
equal to two (there are two random effects) in model 1, the inverse 
Wishart prior for $\bm{\Psi}$, that grants an $IG(\epsilon, \epsilon)$
on the variance parameters, is $IW(1+\epsilon, \epsilon I_2)$.

\begin{itemize}
	\item inverse-Wishart
	
\begin{equation}
p(\bm{\Psi}) \propto IW(\nu, S_{0})
\end{equation}

where we choose $\nu = k-1+e$, and $S_{0}=diag(k-1+e)$, following 
indications by Gelman \textit{et al} (2014). Given a $\bm{\Psi} 
\sim IW(\nu, S_{0})$, where $\bm{\Psi}$ and $S_{0}$ are $k \times 
k$ matrices, it is known that $\bm{\Psi}_{11}$, the $k_1 \times 
k_1$ upper-left triangular sub-matrix of $\bm{\Psi}$, has an 
inverse-Wishart distribution as well. In particular, $\bm{\Psi}_{11} 
\sim IW(\nu-(k-k_1), S_{011})$. Furthermore, for a univariate case 
($k=1$), we know that an inverse Wishart distribution simplifies 
to an inverse Gamma with parameters $\alpha=\frac{\nu}{2}, \beta = 
\frac{S_{0kk}}{2}$. With the goal of resembling as close as possible 
what Gelman 2006 did, we try to define inverse wishart priors for 
$\bm{\Psi}$, such that the the marginal disitbrution is as close as 
possible to the IG(e, e) used by Gelman. This goal is achieved by 
setting $\nu = k-1+e$ and $S_{011} = k-1+e$, which makes the marginal 
distribution on the variance components (diagonals of $\bm{\Psi}$) 
$\sigma_{ii}^{2} \sim IG(\frac{\nu-1}{2},\frac{S_{0ii}}{2})$. The 
inverse-Wishart priors we defined are:

\begin{center}
\begin{tabular}{ c c c }
 Prior Description & $\nu$ & $S_{0}$ \\ 
 \hline
 \multicolumn{1}{l}{1. IW educated} & $2$ & educated guess \\
 \multicolumn{1}{l}{2. IW uninformative} & $k-1+e$ & $(k-1+e-1)\times$ diag(k) \\  
 \hline
\end{tabular}
\end{center}
1, .01, and .001 are then used as values of e.

	\item inverse-Wishart \textit{a là} Huang and Wand

\begin{equation}
	\begin{split}
	p(\bm{\Psi}|a_1, a_2)& \propto IW(\nu + k - 1, 2\nu \times diag(1/a_1, 1/a_2)), \\
	a_k& \propto IG(1/2,1/A_k^2),
	\end{split}
\end{equation}

with $\nu = 2$ and $\bm{A} = [1000, 1000]$. The marginal distribution of any standard deviation term in $\bm{\Psi}$ is Half-\textit{t}$(\nu, A_k)$ and, when choosing $\nu = 2$, the marginal distribution on the correlation term is uniform on (-1, 1), see property 2 to 4 in Huang and Wand (2013, p. 442). Furthermore, according to Huang and Wand (2013, p. 441) arbitrarily large values for $a_k$ lead to arbitrarily weak priors on the standard deviation term. Hence, our choices for the parameters of this prior are: 
	
\begin{center}
\begin{tabular}{ c c c }
 Prior Description & $\nu$ & $\bm{A}$ \\ 
 \hline
 \multicolumn{1}{l}{3. IW a là HW} & $2$ & $[1000, 1000]$ \\  
 \hline
\end{tabular}
\end{center}	
	
	\item Matrix-F variate
	
Following Mulder, Pericchi 2018, I defined a matrix-F variate 
distribution as a prior for the covariance matrix of 	the random
effects $\bm{\Psi}$
	
\begin{equation}
\begin{split}
p(\bm{\Psi})& \propto F(\bm{\Psi}; \nu, \delta, \bm{B}) \\
& \propto \int IW(\bm{\Psi}; \delta + k - 1, \Sigma) \times W(\bm{\Sigma}; \nu, \bm{B})d\bm{\Sigma}
\end{split}
\end{equation}	

with degrees of freedom $\nu>k-1$, $\delta>0$, and $\bm{B}$ a 
positive definite scale matrix that functions as prior guess. 
Different strategies can be followed in trying to achieve vagueness
of this prior. In the literature, the improper prior $(\sigma
^2)^{-\frac{1}{2}}$ has been proposed for the random effects
variance (Berger, 2006; Berger and Strawderman, 1996). % same quotes from Mulder Pericchi 2018
Placing a matrix-F prior on $\bm{\Psi}$, one can approximate
the improper $|\bm{\Psi}|^{-\frac{1}{2}}$ by choosing $\nu=k$,
the smallest allowed integer, $\bm{B}=b\bm{I}_k$ and letting
$b \rightarrow +\infty$ for fixed values of $\delta$.

Another approach might be trying to achieve a flat prior on the
standard deviations of the random effects $p(\sigma) \propto 
1$, through some other proper neighbor definition. Considering
$\bm{\Psi} \sim matrix-F(\nu, \delta, \bm{B})$, it is known that
the marginal distribution of diagonal elements of $\bm{\Psi}$,
the variance components $\sigma^2_{jj}$, is univariate $F(\nu, 
\delta, b_{jj})$ which is equivalent to $p(\sigma_{jj};\nu, \delta, 
b) \propto \sigma^{\nu-1}(1+\sigma^2/b)^{-\frac{\nu+\delta}{2}}$.
By choosing $\nu = k-1+\epsilon$, with say $\epsilon = .001$,
$\delta = 1$, and $b \rightarrow \infty$, $p(\sigma) \propto 1$
is obtained. Hence, the specification $\bm{\Psi} \sim matrix
-F(\nu = 1.001, \delta = 1, \bm{B} = 1e3\bm{I}_2)$ grants a 
proper prior neighbor to a flat prior on the standard deviation
of the random effects.

It is also interesting to approach the vagueness issue by defining
a vague prior weakly centered around an educated guess. Such educated
guess can be obtained by plotting fitted regression lines to all
the clusters in the data set that is under analysis, and trying to
visually assess the intercepts and the slope variance. For example,
consider the data set on depression that will be used to present
the first empirical results. \textit{here add plot and describe how
you arrived to the educated guess}
We can then try to achieve vagueness of the prior distribution by
defining $\nu$ as small as possible. Hence, another prior definition
that was used in this paper, set $\nu = k - 1 + \epsilon, \delta =
1$ and $\bm{B} = matrix(21, 0, 0, 9)$.

Finally, we could also specify a prior that uses the empirical
guess defined by Kass and Natarajan (2006). To do so, we specify
the vague matrix-F prior with scale matrix $\bm{R}^*$, and $\nu 
= 2$.

The following table summarizes the matrix-F specifications that
grant some form of non informativeness and that are compared in 
this paper.

\begin{center}
\begin{tabular}{ c c c c }
 Prior Description & $\nu$ & $\delta$ & $S_{0}$ \\ 
 \hline
 \multicolumn{1}{l}{$|\bm{\Psi}|^{-\frac{1}{2}}$} & $2$ & $1$ & $10^3 \times \bm{I}_2$ \\
 \multicolumn{1}{l}{$p(\sigma)\propto1$} & $1.001$ & $1$ & $10^3 \times \bm{I}_2$ \\
 \multicolumn{1}{l}{vaguely centered around educated guess} & $k-1+\varepsilon$ & $\varepsilon$ & educated guess \\
 \multicolumn{1}{l}{vaguely centered around empirical $R^*$} & $2$ & $1$ & \textbf{\textit{R}}* \\ 
 \hline
  \multicolumn{1}{l}{$\varepsilon$ is set to each 1, and .1} 
\end{tabular}
\end{center}

\end{itemize}

\section{Results}

A good performance is shows by posteriors that give a reasonable 
range for the values of the parameters without having extremely
long right-tails and without having excessively high peaks at zero.

In the figures, the empty circle represents the REML estimates
of the parameter of interest. REML is used becasue it's the
equivalent procedure to RIGLS, one of the most important ML
estimation methods for Generalized Linear Models. RIGLS coincide 
with REML for all Gaussian models such as the one described 
in this article (Goldstein, 2010; Browne Draper, 2006).

When discussing the proper neighbor prior in the matrix F case
for the SD of the variance, you should highlight that it is 
not an ideal situation, the role of a weakly informative prior
is to regularize the posterior distribution so as to keep it into
reasonable bounds (Gelman et al 2014 book). As a general guideline,
priors should be made more precises as posteriors are more vague
(which happens when fewer data are available).

When discussing the IW prior with e you can highlight that, 
compared to the uninformative matrix F priors, the range of 
values supported by the posterior distribution is decidedly  
more concentrated below 2 with a sharp peak towards 0.

In comparison, even when data is scarce (last few conditions),  
the posterior distributions obtained with the matrix F distribution
allow for plausible values in a larger range that is more 
meaningful, while regularized by the prior to avoid impossible
high values (which does happen when using the proper neighbor 
prior).

When discussing the priors for the correlation in the IW and
matrix-F case, you should point out that the beta priors obtained,
are weakly informative according to the symmetry principle: 
symmetrical prior distributions do not pull the posteriors in
any particular direction.

It is also interesting that the results obtained with the $R^*$
priors, both when used in the inverse-Wishart and the matrix-F
are extremely similar to the ones obtained with the educated 
guess obtained by looking at the data.

\section*{Derivation of Full Conditional Posterior Distribution}

The derivation of the conditional posterior follows.

\paragraph{Full conditional for $\bm{\theta}$}(fixed effects) \\

Let's start with
 \begin{equation*}
	p(\bm{\theta}|\bm{y}, \bm{X}, \bm{Z}, \bm{b}_{i}, \bm{\Psi}, \sigma^2) = 	p(\bm{y}|\bm{\theta}, \bm{X}, \bm{Z}, \bm{b}_{i}, \bm{\Psi}, \sigma^2) p(\bm{\theta})
 \end{equation*}
where
 \begin{equation*}
  \begin{split}
  p(\bm{y}|\bm{\theta},\bm{X},\bm{Z},\bm{\Psi},\sigma^2) &= \prod_{i=1}^n \prod_{j=1}^Jp(y_{ij}|\bm{\theta}^{T}\bm{x}_{ij} + \bm{b}_{i}^{T}\bm{z}_{ij}, \bm{\Psi}, \sigma^2) \\
  &\propto exp(-\frac{1}{2\sigma^2}SSR)
  \end{split}
 \end{equation*}
and
 \begin{equation*}
  SSR = \sum_{i = 1}^{n}\sum_{j = 1}^{J}( y_{ij}-\bm{\theta}^{T}\bm{x}_{ij} - \bm{b}_{i}^{T}\bm{z}_{ij})^2
 \end{equation*}
where can rewrite $y_{ij}$ as $\tilde{y}_{ij}$, with $\tilde{y}_{ij} = y_{ij} - \bm{b}^{T}_{i}\bm{z}_{ij}$ which makes SSR:
 \begin{equation*}
  \begin{split}
   SSR& = \sum_{i = 1}^{n}\sum_{j = 1}^{J}( \tilde{y}_{ij}-\bm{\theta}^{T}\bm{x}_{ij})^2\\
   &= ( \tilde{\bm{y}} - \bm{X}\bm{\theta} )^{T}( \tilde{\bm{y}} - \bm{X}\bm{\theta} )\\
   &= \tilde{\bm{y}}^{T}\tilde{\bm{y}} - 2\bm{\theta}^{T}\bm{X}\tilde{\bm{y}} + \bm{\theta}^{T}\bm{X}^{T}\bm{X}\bm{\theta}
  \end{split}
 \end{equation*}
Hence,
 \begin{equation*}
	p(\bm{y}|\bm{\theta}, \bm{X}, \bm{Z}, \bm{\Psi}, \sigma^2) \propto exp(-\frac{1}{2\sigma^2}[- 2\bm{\theta}^{T}\bm{X}\tilde{\bm{y}} + \bm{\theta}^{T}\bm{X}^{T}\bm{X}\bm{\theta}])
 \end{equation*}
Combining this with the prior we obtain:
 \begin{equation}
  \begin{split}
	p(\bm{\theta}|\bm{y}, \bm{X}, \bm{Z}, \bm{\Psi}, \sigma^2)& \propto exp(-\frac{1}{2}\bm{\theta}^{T}\bm{X}^{T}\bm{X}\bm{\theta} + \bm{\theta}^{T}\bm{X}\tilde{\bm{y}}) \\
	\bm{\theta}|.&\sim \bm{N}\left(\frac{(\bm{X}^{T}\bm{X})^{-1}\bm{X}\bm{\tilde{y}}}{\sigma^2}, \frac{(\bm{X}^{T}\bm{X})^{-1}}{\sigma^2}\right)
  \end{split}
 \end{equation}

\paragraph{Full conditional for $\bm{b}_{i}$} (random effects)\\

To derive this one we can start from:
 \begin{equation*}
	p(\bm{b}_{i}|\bm{y}, \bm{X}, \bm{Z}, \bm{\theta}, \bm{\Psi}, \sigma^2) = 	p(\bm{y_{i}}|\bm{\theta}, \bm{b}_{i}, \bm{X}, \bm{Z}, \bm{\Psi}, \sigma^2) p(\bm{b}_{i})
 \end{equation*}
We know that
 \begin{equation*}
	p(\bm{y}_{i}|.) = \prod_{j=1}^{J}p(y_{ij}|\bm{\theta}^{T}\bm{x}_{ij} + \bm{b}_{i}^{T}\bm{z}_{ij}, \sigma^2) \propto exp(-\frac{1}{2\sigma^2}SSR_{i})
 \end{equation*}
with
 \begin{equation*}
	SSR = \sum_{j = 1}^{J}( y_{ij}-\bm{\theta}^{T}\bm{x}_{ij} - \bm{b}_{i}^{T}\bm{z}_{ij})^2
 \end{equation*}
and we can rewrite $y_{ij}$ as $\tilde{y}_{ij} = y_{ij} - \bm{\theta}^{T}\bm{x}_{ij}$, which would make SSR be\\
 \begin{equation*}
  \begin{split}
	SSR& = \sum_{j = 1}^{J}( \tilde{y}_{j}-\bm{\theta}^{T}\bm{x}_{j})^2\\
	&= ( \tilde{\bm{y}} - \bm{b}_{i}^{T}\bm{Z}_{i})^{T}( \tilde{\bm{y}} - \bm{b}_{i}^{T}\bm{Z}_{i})\\
	&= \tilde{\bm{y}}^{T}\tilde{\bm{y}} - 2\bm{b}_{i}^{T}\bm{Z}_{j}\tilde{\bm{y}} + \bm{b}_{i}^{T}\bm{Z}^{T}_{i} \bm{Z}_{j}\bm{b}_{i}
  \end{split}
 \end{equation*}
Hence,
 \begin{equation*}
	p(\bm{y}_{i}|.) \propto exp(-\frac{1}{2\sigma^2}[- 2\bm{b}_{i}^{T}\bm{Z}_{j}\tilde{\bm{y}} + \bm{b}_{i}^{T}\bm{Z}^{T}_{i} \bm{Z}_{j}\bm{b}_{i}])
 \end{equation*}
We also know that in this case, the "prior" is
 \begin{equation*}
	p(\bm{b}_{i}) \propto N(\bm{0}, \bm{\Psi}) \propto exp(-\frac{1}{2}[-2\bm{b}_{i}^{T}\bm{\Psi}^{-1}\bm{0} + \bm{b}_{i}^{T}\bm{\Psi}^{-1}\bm{b}_{i}])
 \end{equation*}
In conclusion, combining the sampling model and the prior, we get:
 \begin{equation}
  \begin{split}
	p(\bm{b}_{i}|.)& \propto 	
	exp(
	-\frac{1}{2\sigma^2}[- 2\bm{b}_{i}^{T}\bm{Z}_{j}\tilde{\bm{y}} + \bm{b}_{i}^{T}\bm{Z}^{T}_{i} \bm{Z}_{j}\bm{b}_{i}]
	-\frac{1}{2\sigma^2}[- 2\bm{b}_{i}^{T}\bm{Z}_{j}\tilde{\bm{y}} + \bm{b}_{i}^{T}\bm{Z}^{T}_{i} \bm{Z}_{j}\bm{b}_{i}])\\
	\bm{b}_{i}|.& \propto \bm{N}\left(\left(\Psi^{-1} + \frac{\bm{Z}_{i}^{T}\bm{Z}_{i}}{\sigma^2}\right)^{-1}\left(\bm{\Psi}^{-1}\bm{0}+\frac{\bm{Z}_{i}^{T}\tilde{y}_{i}}{\sigma^2}\right), \left(\Psi^{-1} + \frac{\bm{Z}_{i}^{T}\bm{Z}_{i}}{\sigma^2}\right)^{-1}\right)
  \end{split}
 \end{equation}

\paragraph{Full conditional for $\sigma^2$}(error variance)\\

The full conditional posterior can be expressed as:
 \begin{equation*}
	p(\sigma^2|\bm{y}, \bm{X}, \bm{Z}, \bm{\theta}, \bm{b}_{i}, \bm{\Psi}) = 	p(\bm{y}|\bm{\theta}, \bm{b}_{i}, \bm{X}, \bm{Z}, \bm{\Psi}, \sigma^2) p(\sigma^2)
 \end{equation*}
The sampling model is the same we saw for the full conditional distribution of $\bm{\theta}$:
 \begin{equation*}
  \begin{split}
	p(\bm{y}|\bm{\theta}, \bm{X}, \bm{Z}, \bm{\Psi}, \sigma^2)& = \prod_{i=1}^n \prod_{j=1}^Jp(y_{ij}|\bm{\theta}^{T}\bm{x}_{ij} + \bm{b}_{i}^{T}\bm{z}_{ij}, \bm{\Psi}, \sigma^2)\\
	&= \prod_{i=1}^{n} \prod_{j=1}^{J}(2\pi\sigma^{-2})^{-\frac{1}{2}}exp(-\frac{(y_{ij} - \bm{\theta}^{T}\bm{x}_{ij} - \bm{b}_{i}^{T}\bm{z}_{ij})^2}{2\sigma^2})	
  \end{split}
 \end{equation*}
However, we are now interested in $\sigma^2$, hence
 \begin{equation*}
  \begin{split}
	p(\bm{y}|\bm{\theta},\bm{X},\bm{Z},\bm{\Psi},\sigma^2)& \propto (\sigma^{2})^{-\frac{N}{2}}exp(-\frac{\sum_{i = 1}^{n}\sum_{j = 1}^{J}( y_{ij}-\bm{\theta}^{T}\bm{x}_{ij} - \bm{b}_{i}^{T}\bm{z}_{ij})^2 }{2\sigma^2}) \\
	&\propto (\sigma^{2})^{-\frac{N}{2}}exp(-\frac{1}{2\sigma^2}SSR)
  \end{split}
 \end{equation*}
where $N = \sum_{i}^{n}nj_{i}$ is the entire sample size (all observations within all clusters).
The prior for $\sigma$ is given above, and therefore we can write the full conditional posterior as:
 \begin{equation}
  \begin{split}
	p(\sigma^2|\bm{y},\bm{X},\bm{Z},\bm{\theta},\bm{b}_{i},\bm{\Psi})& \propto (\sigma^{2})^{-\frac{N}{2}-1}exp(-\frac{1}{2\sigma^2}SSR)\\
	\sigma^2|.& \sim IG(\frac{N}{2}, \frac{SSR}{2})
  \end{split}
 \end{equation}

\paragraph{Full conditional for $\bm{\Psi}$}(random effects variance covariance matrix)\\

Here, we need to write down the posteriors for the different priors we specified. First, let us define the sampling model for the random effects.
 \begin{equation}
  \begin{split} 
   \begin{bmatrix} 
	   b_{0i}\\ 
	   b_{1i}
   \end{bmatrix} 
  	= \bm{b}_{i}& \sim N(\bm{0}, \bm{\Psi}) \\
 	p(\bm{b}_{1}, \bm{b}_{2}| \bm{\Psi})& \propto |\bm{\Psi}|^{-\frac{n}{2}} exp\left(-\frac{1}{2}tr(\bm{S}_b\bm{\Psi}^{-1})\right)  
  \end{split} 
 \end{equation}
where $\bm{S}_b$ is $\Sigma_i\bm{b}_i\bm{b}_i^{T}$
\begin{itemize}
	\item given the inverse-Wishart prior

\begin{equation*}
 \begin{split} 
  p(\bm{\Psi})& \propto IW(\nu, \bm{S}_{0}) \\
  & \propto |\bm{\Psi}|^{-\frac{(\nu + k + 1)}{2}}exp\left(-\frac{1}{2}tr(\bm{S}_0\bm{\Psi}^{-1}) \right)
 \end{split}
\end{equation*}

the full conditional posterior of $\bm{\Psi}$ is

\begin{equation}
 \begin{split} 
  p(\bm{\Psi}|.)& \propto |\bm{\Psi}|^{-\frac{(\nu + n + k + 1)}{2}}exp\left(-\frac{1}{2}tr([\bm{S}_0+\bm{S}_b]\bm{\Psi}^{-1}) \right) \\
  & \propto IW(\nu + n, \bm{S}_{0} + \bm{S}_{b})
 \end{split}
\end{equation}	

where $\nu = 2$

	\item inverse-Wishart \textit{a là} Huang and Wand
	
\begin{equation*}
	\begin{split}
	p(\bm{\Psi}|a_1, a_2)& \propto IW(\nu + k - 1, 2\nu diag(1/a_1, 1/a_2)), \\
	a_k& \propto IG(\eta,1/A_k^2) \\
	p(\bm{\Psi})& \propto |\bm{\Psi}|^{-\frac{(\nu+k-1+1)}{2}}exp\left(-\frac{1}{2}tr(2\nu diag(1/a_1, 1/a_2)\bm{\Psi}^{-1}) \right) \\
	& \times \left(\frac{1}{a_1}\right)^{\eta+1}exp\left(-\frac{1}{A_1^2a_1}\right) \times \left(\frac{1}{a_2}\right)^{\eta+1}exp\left(-\frac{1}{A_2^2a_2}\right) \\
	\end{split}
\end{equation*}

the full conditional posterior of $\bm{\Psi}$ is

\begin{equation}
 \begin{split} 
  p(\bm{\Psi}|.)& \propto |\bm{\Psi}|^{-\frac{(\nu+k-1+n+1)}{2}}exp\left(-\frac{1}{2}tr([\bm{S}_b+2\nu diag(1/a_1, 1/a_2)]\bm{\Psi}^{-1}) \right) \\
  & \propto IW(\nu+k-1+n, \bm{S}_{b}+2\nu diag(1/a_1, 1/a_2))\\
  p(a_k|.)& \propto IG\left(\eta(\nu+k), \nu\left(\bm{\Psi}^{-1}_{kk}+\frac{1}{A_k^2}\right)\right)
 \end{split}
\end{equation}		

where $\eta = \frac{1}{2}, \nu = 2, k = 2$, and $n$ is the number of clusters (individuals). (For the conditional posterior of $a_k$ refer to Huang and Wand (2013), section 4.2).
	
	\item Matrix-F variate

Following section 2.3 in Mulder and Pericchi (2018), instead of working directly with the $\bm{\Psi} \sim F(\nu, \delta, \bm{B})$ we apply the parameter expansion defined above (see section on priors) and model it as $\bm{\Psi} \sim IW(\delta + k - 1, \bm{\Omega})$ with $\bm{\Omega} \sim W(\nu, \bm{B})$. With this parameter expansion, the conditional priors are: 
\begin{equation*}
 \begin{split} 
  \bm{\Psi|\Omega}& \sim IW(\delta+k-1, \bm{\Omega}) \\
  \bm{\Omega|\Psi}& \sim W(\nu+\delta+k-1, (\bm{\Psi}^{-1} + \bm{B}^{-1})^{-1}) \\
 \end{split}
\end{equation*}
% Old workout of matrix F (keep for a bit)
%\begin{equation*}
% \begin{split} 
%  p(\bm{\Psi})& \propto F(\bm{\Psi}; \nu, \delta, \bm{B}) \\
%  & \propto \int IW(\bm{\Psi}; \delta + k - 1, \bm{\Omega}) \times W(\bm{\Omega}; \nu, \bm{B})d\bm{\Omega} \\
%  & \propto \int |\bm{\Psi}|^{-\frac{(\delta+2k)}{2}}exp\left(-\frac{1}{2}tr(\bm{\Omega}\bm{\Psi}^{-1})\right) \times |\bm{\Omega}|^{-\frac{(\nu-k-1)}{2}}exp\left(-\frac{1}{2}tr(\bm{\Omega}\bm{B}^{-1})\right)d\bm{\Omega}
% \end{split}
%\end{equation*}

which makes the full conditional posterior of:
\begin{equation*}
 \begin{split} 
  \bm{\Psi|\Omega},.& \sim IW(\delta+k-1+n, \bm{S}_b + \bm{\Omega}) \\
  \bm{\Omega|\Psi},.& \sim W(\nu+\delta+k-1, (\bm{\Psi}^{-1} + \bm{B}^{-1})^{-1}) \\
 \end{split}
\end{equation*}
% Old workout of matrix F (keep for a bit)
%\begin{equation}
% \begin{split} 
%  p(\bm{\Psi}|.)& \propto \int |\bm{\Psi}|^{-\frac{(\delta+2k+n)}{2}}exp\left(-\frac{1}{2}tr([\bm{\Omega}+\bm{S}_b]\bm{\Psi}^{-1})\right) \times |\bm{\Omega}|^{-\frac{(\nu-k-1)}{2}}exp\left(-\frac{1}{2}tr(\bm{\Omega}\bm{B}^{-1})\right)d\bm{\Omega} \\
%  & \propto \int IW(\delta+k+n-1, \bm{\Omega} + \bm{S}_{b}) \times W(\nu, \bm{B})d\bm{\Omega}
% \end{split}
%\end{equation}			

with parameters as defined above. Given these posteriors, the Gibbs sampler implementation is straightforward.
	
\end{itemize}

\paragraph{Notation Conventions}

\begin{itemize}
	\item $n$ number of clusters; $i$ specific cluster
	\item $J$ number of observations within cluster; $j$ specific observation
	\item $N$ total number of observations
\end{itemize}



\end{document}


